\chapter{Wstęp}

\section{Cele projektu}

Głównym celem projektu jest stworzenie aplikacji, która będzie umożliwiała wpisanie przez użytkownika dowolnego tekstu w języku polskim oraz pozwoli na jego sprawdzenie pod względem użycia poprawnych słów. Następnie użytkownik będzie miał możliwość wstawienia za każde z wychwyconych błędnych wyrazów, innego, jednego z tych, które będą sugerowane przez aplikację, bądź też zignorowanie sugestii. Aplikacja wyposażona będzie w algorytm sprawdzający czy wyraz istnieje w słowniku języka polskiego, gdy takiego nie znajdzie, wyszukiwane będą słowa o podobnej budowie, lecz nie koniecznie pasujące znaczeniowo do kontekstu zdania.

\section{Założenia projektowe}

W projekcie wykorzystany został język programowania C\#, z wykorzystaniem technologii WPF, umożliwiającej tworzenie graficznego interfejsu użytkownika w oparciu o składnię XAML.

W projekcie wykorzystany został również słownik języka polskiego dostępny na stronie słownika języka polskiego [\ref{bib:slownik}]. Słownik ten zawiera podstawowe formy słów wraz ze wszystkimi możliwymi odmianami. Słownik ten przewiduje jedynie słowa o maksymalnej długości wynoszącej 15 znaków i nie zawiera żadnych nazw własnych tzn. imion, nazw krajów, miast itp.

\section{Zakres projektu}
Zakres projektu dotyczy zaprojektowania i implementacji aplikacji umożliwiającej pisanie własnych tekstów, które na bieżąco są sprawdzane pod względem poprawności ortograficznej.

Umożliwione użytkownikowi zostało również wklejanie gotowych tekstów, w celu ich sprawdzenia.

Projekt został wyposażony w trzy algorytmy umożliwiające uzyskanie podpowiedzi dla użytkownika oraz opracowane zostały sposoby optymalizacji tak, aby aplikacja działała w czasie rzeczywistym mimo słownika, który zawiera prawie 3 miliony pozycji.

\section{Opis rozdziałów}
\begin{enumerate}
	\item Rozdział I – opisuje założenia wstępne oraz zawiera informacje, na temat tego co projekt powinien zawierać. Rozdział ten ponadto stanowi wstęp, w którym zawarte są opisy wszystkich rozdziałów niniejszego dokumentu.
	\item Rozdział II – zawiera opis użytych w projekcie algorytmów wraz z diagramami.
	\item Rozdział III – opisuje fazę  projektową oraz szczegóły implementacyjne projektu oraz opis użytych metod przyśpieszających działanie programu.
	\item Rozdział IV – podsumowanie projektu wraz z analizą osiągniętych celów.
\end{enumerate}




