\chapter{Podsumowanie}

Aplikacja stworzona na potrzeby projektu spełnia wszystkie cele oraz założenia projektowe. Pozwala ona na sprawdzenie dowolnego tekstu w języku polskim wpisanego przez użytkownika. Użytkownik ma możliwość wstawienia w każde z błędnych słów innego zasugerowanego przez system. Dzięki zoptymalizowaniu algorytmów wyszukiwania jak i interfejsu użytkownika, wyszukiwanie sugestii odbywa się w czasie rzeczywistym. Dzięki zastosowaniu słownika w postaci pliku tekstowego jego modyfikacja w formie dodawania kolejnych wyrazów jest bardzo prosta. \\ 

W celu wyszukiwania jak najtrafniejszych zamienników dla błędnie napisanych słów zaimplementowano trzy algorytmy:
\begin{itemize}
	\item Odległość Levenshteina - oblicza on odległość edycyjną dwóch porównywanych napisów, odległość ta wyrażana jest przez liczbę operacji podstawowych. Jednak konieczność przeglądania całego słownika, w dość znacznym stopniu obciąża procesor, dlatego też w projekcie zastosowano wyłącznie porównywanie słów o podobnej długości (plus, minus jeden znak).
	\item Zamiana znaków - algorytm ten zamienia ciągi znakowe odpowiadające częstym błędom ortograficznym w języku polskim i na tej podstawie utworzyć może wiele kombinacji i w pewnym stopniu uzupełnia uproszczenie zastosowane w przypadku użycia odległości Levenstheina dla wybranych fragmentów słownika.
	\item Podział na wyrazy - algorytm ten przeciwdziała częstym błędom w piśmie komputerowym, wynikającym z braku wciśnięcia spacji. 
\end{itemize}

Dzięki takiemu rozwiązaniu program dobrze radzi sobie niezależnie od rodzaju błędu jaki popełnił użytkownik.

Aplikacja została napisana w języku C\#, dlatego jest przeznaczona na komputery z systemem Windows. Korzystanie z niej nie wymaga instalacji. Dodatkowo jest to niezależna aplikacja, która do działania nie wymaga dodatkowych bibliotek.